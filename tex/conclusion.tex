\chapter*{Заключение}
\addcontentsline{toc}{chapter}{Заключение}

В ходе выполнения данной работы были классифицированы методы внесения изменений в ядро операционной системы Linux. 

На основании результатов сравнения методов можно сделать вывод о том, что наиболее эффективным методом является динамический метод, так как не требует перезагрузки системы, решает проблему появления времени простоя и не является ресурсозатратным.

Цель, поставленная в начале работы была достигнута. В ходе ее выполнения были решены следующие задачи:

\begin{itemize}
	\item были изучены существующие методы внесения изменений в ядро Linux;
	\item были выделены критерии оценки изученных методов;
	\item было проведено сравнение методов на основании выделенных критериев;
	\item были отражены результаты сравнения в выводе.
\end{itemize}